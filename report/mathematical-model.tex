\section{Mathematical Model}
\label{sec:mathmod}

To model these systems \textit{in silico}, an appropriate mathematical model must first be derived from Maxwell's Equations. The development of said model is arranged as follows. Section \ref{subsub:eie} contains the derivation of the EIE from Maxwell's Equations followed by surface element discretization into a simultaneous set of matrix equations. Section \ref{subsub:mat-form} outlines three formulations of the system matrix $[A]$ starting with square elements approximated as circles, an exact solution for square elements, and finally subdomain collocation. 

\subsubsection{The Electrostatic Integral Equation (EIE)}
\label{subsub:eie}
From basic electrostatics, the integral form of Gauss's Equation over some surface $S$ is as follows \cite{rothlecnotes}, \cite{jin2011theory}
\begin{align}
    \iint_S\epsilon_0^{-1}g(\textbf{r},\textbf{r'})\rho_s(\textbf{r'})dS'=\Phi(r)
    \label{eq:Gauss}
\end{align}
where $\epsilon_0$ is the permittivity of free space, $\rho_s(\textbf{r'})$ is the surface charge density, $\Phi(r)$ is the electrostatic potential, and $g(\textbf{r},\textbf{r'})$ is the corresponding Green's function (integration kernel)
\begin{align}
    g(\textbf{r},\textbf{r'})=\frac{1}{4\pi|\textbf{r}-\textbf{r'}|}.
    \label{eq:greemn}
\end{align}

Basic electrostatic theory is also used to enforce a boundary condition sufficiently conductive metal (as assumed in this problem). This boundary condition requires the surface of the plate to be at an equipotential, thus $\Phi(r)=\Phi_0$ where $\Phi_0$ is an arbitrary electric equipotential. Under these stipulations, the only unknown in this problem is the charge density on the plate surface. To solve, we write the charge density in terms of a series of linearly independent basis functions, $v_n(\textbf{r'})$, and expansion coefficients, $c_n$, defined over individual elements as follows \cite{rothlecnotes}, \cite{jin2011theory}
\begin{align}
    \rho_s(\textbf{r'})=\sum_{n=1}^{N}c_nv_n(\textbf{r'}).
    \label{eq:rho_expansion}
\end{align}


With the surface charge density discretized, it is now necessary to sample the underlying integral equations in a similar fashion using testing functions $w_m(\textbf{r})$. With these definitions, (\ref{eq:Gauss}) can now be expressed as the EIE which is defined as follows \cite{rothlecnotes}, \cite{jin2011theory}
\begin{multline}
    \sum_{n=1}^{N}c_n\iint_S\iint_Sw_m(\textbf{r})\epsilon_0^{-1}g(\textbf{r},\textbf{r'})v_n(\textbf{r'})dS'dS \\ =\iint_Sw_m(\textbf{r})\Phi_0dS.
    \label{eq:eie}
\end{multline}

Equation (\ref{eq:eie}) can be expressed as the following matrix equation
\begin{align}
    [A]\{c\}=\{b\}
    \label{eq:linsys}
\end{align}
where $A$ is the system matrix defined as
\begin{align}
    [A]_{mn}=\iint_S\iint_Sw_m(\textbf{r})\epsilon_0^{-1}g(\textbf{r},\textbf{r'})v_n(\textbf{r'})dS'dS,
    \label{eq:a}
\end{align}
$b$ contains the equipotential boundary condition
\begin{align}
    \{b\}_m=\iint_Sw_m(\textbf{r})\Phi_0dS,
    \label{eq:b}
\end{align}
and ${c}$ is an array of charge density scalars corresponding to their indexed element which is solved for. The following section will go over several methodologies for defining the system matrix and equipotential boundary condition array \cite{rothlecnotes}, \cite{jin2011theory}.

\subsubsection{System Matrix Formulations}
\label{subsub:mat-form}

For simplicity only the following zeroth order basis function will be used for all analyses
\begin{align}
    v_n(\textbf{r'})=\begin{cases}
        1,\quad r'\in S_n\\
        0,\quad \mathrm{else}
    \end{cases}
\end{align}
where $S_n$ is defined as the area of the square surface element $n$.

The first method that will be investigated involves using a Dirac delta function as the testing function
\begin{align}
    w_m(\textbf{r})=\delta (\textbf{r}-\textbf{r}_m)
    \label{eq:Dirac}
\end{align} 
where $\textbf{r}_m$ denotes the center of element $m$. This testing function can be interpreted as testing each element at it's respective center point only \cite{rothlecnotes} This testing function will be used for the first two methods discussed.

Using (\ref{eq:Dirac}), (\ref{eq:a})-(\ref{eq:b}) can be re-written as
\begin{align}
    [A]_{mn}=\iint_{S_{n}} \epsilon_0^{-1}g(\textbf{r},\textbf{r'})dS',
    \label{eq:a2}
\end{align}
and 
\begin{align}
    \{b\}_m=\Phi_0.
    \label{eq:b2}
\end{align}

This leads to the first two formulations of the system matrix $A$. In the first case, (\ref{eq:a2}) can be approximated using midpoint integration for all $m \neq n$. The case containing $m=n$ contains a numerical singularity. To extract this singularity, the element can be approximated as a circular element possessing the same area as $S_n$. This results in the following formulation of the system matrix \cite{rothlecnotes}, \cite{jin2011theory},
\begin{align}
    [A]_{mn}=\begin{cases}
        \frac{S_n}{4\pi\epsilon_0|\textbf{r}_m-\textbf{r}_n|}, \quad m\neq n \\
        \frac{1}{2\epsilon_0}\sqrt{\frac{S_n}{\pi}}, \quad m=n.
    \end{cases}
    \label{eq:aaprox}
\end{align}
For the duration of this paper, this method will be referred to as the circular element approximation.

The next methodology directly integrates (\ref{eq:a2}) for square elements thus eliminating the need to use an approximation. The closed form of (\ref{eq:a2}) for square elements is
\begin{multline}
    [A]_{mn}=\frac{1}{4\pi\epsilon_0}\big((x_m-x')\ln((y_m-y')+R)+\\
    (y_m-y')\ln((x_m-x')+R)\big)
    \label{eq:aexact}
\end{multline}
where $R=\sqrt{(x_m-x')^2+(y_m-y')^2}$ evaluated over $x'=x_n\pm\Delta x / 2$, $y'=y_n\pm\Delta y / 2$ \cite{jin2011theory}. This method will be referred to as the exact square element method.

For the third and final method studied here, the pulse function is used as a basis function as opposed to the Dirac delta function, (\ref{eq:Dirac}), used in (\ref{eq:a2}). This is physically equivalent to averaging the testing function over the entire element as opposed to measuring only at the element center point as in the first two methods \cite{rothlecnotes}, \cite{jin2011theory}. This methodology gives rise to the following system matrix
\begin{align}
    [A]_{mn}=\iint_{S_m}\iint_{S_n} \epsilon_0^{-1}g(\textbf{r},\textbf{r'})dS'dS,
    \label{eq:a3}
\end{align}
and boundary condition array
\begin{align}
    \{b\}_m=S_m\Phi_0.
    \label{eq:b3}
\end{align}

Under this formulation, (\ref{eq:a3}) can be exactly solved resulting in
\begin{multline}
    [A]_{mn}=\frac{1}{4\pi\epsilon}\Big(\frac{(x-x')^2(y-y')}{2}\ln((y-y')+R) \\ +\frac{(x-x')(y-y')^2}{2}\ln((x-x')+R)\\-\frac{(x-x')(y-y')}{4}((x-x')+(y-y')) -\frac{R^3}{6}\Big)
\end{multline}
where $R=\sqrt{(x-x')^2+(y-y')^2}$ evaluated over $x'=x_n\pm\Delta x / 2$, $y'=y_n\pm\Delta y / 2 x=x_m\pm\Delta x / 2$, $y=y_m\pm\Delta y / 2$ \cite{jin2011theory}. This method is referred to as subdomain collocation. Unlike the previous set of equations, this method results in symmetric matrices thereby allowing a halving of the total memory footprint (assuming appropriate data structure and solver choice) as only the upper diagonal matrix needs to be stored. Despite this, solving these systems is still asymptotically $O(n^3)$.

One major issue with this formulation is the fact that indeterminate terms ($\ln(0)\times 0$) arise when $x=x'$ or $y=y'$. A trivial means of resolving these indeterminate forms is to add a very small constant $c$ to each $\ln(x+c)$ call such that $c$ is much less than $x$ for any value of $x$ used in the assembly process. Firstly, this prevents indeterminate forms from arising as the natural log of a small number multiplied by zero evaluates to zero. Secondly, provided that $c<<x$, the error introduced by including $c$ can be reduced to machine precision for sufficiently small $c$. Since the introduced error can be made arbitrarily small, there is no need to conditionally check for problematic arguments to $\ln()$ calls as this $c$ term can always be included. This increases assembly efficiency as the matrix assembler does not need to conditionally check for problematic arguments 16 times for all $n^2$ elements in the system matrix. For the purposes of this work, $c=1\times10^{-100}$ was chosen as it satisfies $c<<x$.