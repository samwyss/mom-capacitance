\section{Mathematical Model}
\label{sec:mathmod}

To model these systems \textit{in silico}, an appropriate mathematical model must first be derived from Maxwell's Equations. The development of said model is arranged as follows. Section \ref{subsub:eie} contains the derivation of the EIE from Maxwell's Equations followed by surface element discretization into a simultaneous set of equations. Section \ref{subsub:mat-approx} outlines three formulations of the Systems matrix $A$ starting with elements approximated as circles, an exact solution, and finally subdomain collocation. 

\subsubsection{The Electrostatic Integral Equation (EIE)}
\label{subsub:eie}
From basic electrostatics, the integral form of Gauss's Equation over some surface $S$ is as follows
\begin{align}
    \iint_S\epsilon_0^{-1}g(\textbf{r},\textbf{r'})\rho_s(\textbf{r'})dS'=\Phi(r)
\end{align}
where $\epsilon_0$ is the permittivity of free space, $\rho_s(\textbf{r'})$ is the surface charge density, $\Phi(r)$ is the electrostatic potential, and $g(\textbf{r},\textbf{r'})$ is the corresponding Green's function (integration kernel)
\begin{align}
    g(\textbf{r},\textbf{r'})=\frac{1}{4\pi|\textbf{r}-\textbf{r'}|}
\end{align}

\subsubsection{System Matrix Approximations}
\label{subsub:mat-approx}




