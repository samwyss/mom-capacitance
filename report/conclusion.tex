\section{Conclusion}
\label{sec:conclusion}
A 2-dimensional method of moments program was developed from Maxwell's Equations allowing for the charge distribution to be calculated on an arbitrarily sized square metallic plate for an arbitrary number of elements per side. The charge distribution was then integrated to find the total charge collected on the metallic plate. Finally, the total charge was then used to solve for the capacitance of the plate given an arbitrary applied equipotential. These predicted capacitances were next verified against those obtained from Monte Carlo methods found in the literature of which they agreed well. The charge distribution as a function of space was then analyzed and explained with basic electrostatic theory. Finally, a convergence study was performed and showed that all methodologies tested converge to a predicted capacitance of approximately $0.4075$pF and all predict values within $0.1\%$ of eachother for $n\geq30$.

While relatively general in the sense that the model works for any side length, applied potential, and number of elements per side, this model is not performant. Upon software profiling, it was discovered that the code spends the majority of its run time assembling the matrices. This is most notable in the exact square element, and subdomain collocation procedures where each element of the system matrix takes between $4-16$ iterations to fully assemble. When combined with the Python's abysmal loop performance, the matrix assembly dominates the program execution time. To that end, future work should first focus rewriting the model in a language like C/C++/Rust, all of which have optimizing compilers that can take advantage of loop unrolling to dramatically speed up the assembly process. After which, the next logical step would be to expand support to handle unstructured meshes such that the capacitance of arbitrary shaped surfaces could be obtained.