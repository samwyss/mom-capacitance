\section{Introduction}
\label{sec:intro}

Contrary to many other methods in Computational Electromagnetics (CEM), the Method of Moments (MoM) is purely based on the integral form of Maxwell's Equations and underlying Green's Functions \cite{rothlecnotes}, \cite{jin2011theory}. In the context of the Electrostatic Integral Equation (EIE), MoM uses the potential contributed by individual point charges on surface elements to solve for charge density as a function of location. This charge distribution can trivially be integrated over surface elements to solve for the net charge on an surface for an arbitrary geometry and applied potential. From this, the capacitance of said structure can easily be obtained.

In matrix form, the EIE relates the point source response of charges in a given surface element to all other surface elements and the applied potential. The use of Green's Functions in the problem formulation allows for these problems to be solved without the need to spatially terminate the simulation region which is a common source of error in other CEM methods. However, the use of Green's Functions results in fully dense matrices, thus requiring the computational complexities and memory footprint that comes with solving them which at first may be seen as a downside of MoM. However, these matrices only require the discretization of surfaces \cite{rothlecnotes}, \cite{jin2011theory} thereby resulting in much smaller system matrices overall which helps to compensate for the typical $O(n^3)$ computation complexity required to solve these dense systems. 

The development and results of this work are laid out as follows. Section \ref{sec:mathmod} contains a short derivation of the EIE from Gauss's Law followed by the setup of the EIE matrix equations for a circular approximation of square surface elements, an exact form for square surface elements, and finally the subdomain collocation. Section \ref{sec:numres} contains a verification of the model with a capacitance range found in the literature, an explanation of the plate charge distribution, and finally, an analysis of the asymptotic number of operations required to determine the plate capacitance for all three methods. Finally, Section \ref{sec:conclusion} contains closing remarks regarding the analysis and potential future work.